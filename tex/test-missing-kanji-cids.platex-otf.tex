% -*- coding: utf-8 mode:latex -*-

% 原ノ味フォント 20211103 グリフ名相違の修正確認
% グリフ比較用

\documentclass[dvipdfmx]{jsarticle}
\usepackage{otf}
\usepackage{lmodern}

\title{原ノ味フォント20211103グリフ名相違の修正確認 \\
--- グリフ比較用 ---}
\author{細田 真道}

\begin{document}

\maketitle

\abstract{%
  このファイルは原ノ味フォントのグリフが正しく生成されているか、
  他のフォントと比較して確認するためのものである。
  そのために、同じDVIから、原ノ味フォント埋め込みPDFと
  フォント非埋め込みPDFの双方を生成している。
  小塚フォントを持つAcrobat Readerでフォント非埋め込みPDFを開くと、
  小塚フォントを使った表示を見ることができる。
  同じAcrobat Readerで原ノ味フォント埋め込みPDFを開くと、
  原ノ味フォントを使った表示を見ることができる。
  これにより双方の表示を比較し、
  デザインの差を除いて同じような表示になれば、
  原ノ味フォントのグリフが正しいと考えることができる。

  源ノ明朝2.000ではaj16-kanji.txtに記載されているグリフ名が古いまま更新されず
  AI0-SourceHanSerifのグリフ名と食い違っているものがあった
  (源ノ明朝2.001で修正され食い違いは解消された)。
  また、源ノ角ゴシック2.004でも同様にaj16-kanji.txtのグリフ名が更新されず
  AI0-SourceHanSansのグリフ名と食い違っているものがある。
  原ノ味フォント20211103では源ノ明朝2.000の食い違いの影響を抑えるため
  独自に修正を試みており、
  このファイルはその修正が正しいか確認するためのものである。
  なお、原ノ味フォント202201xxで食い違いが修正された
  源ノ明朝2.001ベースとなったため独自の修正は不要となった。
}

\clearpage

\section{aj16-kanji.txtグリフ名がAI0-SourceHanSerifに存在しないCID}

\input{missing-kanji-cids-mincho.txt}

\clearpage

\section{aj16-kanji.txtグリフ名がAI0-SourceHanSansに存在しないCID}

\gtfamily
\input{missing-kanji-cids-gothic.txt}

\clearpage

\section{原ノ味明朝20211103との比較用}

\normalsize\mcfamily
次ページ以降は、
原ノ味明朝20211103当時の本ファイルとの比較用である。
つまりベースとしていた源ノ明朝2.000において
aj16-kanji.txtグリフ名がAI0-SourceHanSerifに存在しなかったCIDと、
当時の源ノ明朝JPにグリフが存在しなかったため
原ノ味明朝20211103から欠けてしまっていたCID+13729, CID+14019を示している。
原ノ味明朝20211103ではCID+13729, CID+14019はダミーグリフになっていたが、
原ノ味明朝202201xxではグリフが復活している。

\clearpage

\mcfamily
\input{missing-kanji-cids-mincho-20211103.txt}

\end{document}
