% -*- coding: utf-8 mode:latex -*-

% 原ノ味フォント 20200612 cmap CMap 不一致グリフ一覧
% グリフ比較用

\documentclass[dvipdfmx]{jsarticle}
\usepackage{otf}
\usepackage{lmodern}
\usepackage{xcolor}

\title{原ノ味フォント20200612 CMap cmap不一致グリフ一覧 \\
--- グリフ比較用 ---}
\author{細田 真道}

\begin{document}

\maketitle

\abstract{%
  このファイルはAJ1 CMapと、原ノ味フォント生成途中に得られたcmapテーブルで、
  同じUnicodeコードポイントから異なるCIDへマッピングされたグリフを集め、
  他のフォントと比較して確認するためのものである。
  そのために、同じDVIから、原ノ味フォント埋め込みPDFと
  フォント非埋め込みPDFの双方を生成している。
  小塚フォントを持つAcrobat Readerでフォント非埋め込みPDFを開くと、
  小塚フォントを使った表示を見ることができる。
  同じAcrobat Readerで原ノ味フォント埋め込みPDFを開くと、
  原ノ味フォントを使った表示を見ることができる。
  これにより双方の表示を比較し、
  デザインの差を除いて同じような表示になれば、
  原ノ味フォントのグリフは正しいがcmapテーブルには問題があり、
  cmapテーブルをCMapで上書きすべきであると考えることができる。
}

\clearpage
\parindent=0pt
\fboxsep=0pt

\definecolor{gray1}{rgb}{.9,.9,.9}
\newcommand{\test}[1]{{\Huge\colorbox{gray1}{#1}\textmd{~}}}

\mcfamily
{\Large 明朝}

U+00AD→CMap CID+151, cmap CID+14

\test{\CID{151}} \test{\CID{14}}

U+2EB3→CMap CID+14189, cmap CID+18384

\test{\CID{14189}} \test{\CID{18384}} 

U+2EC1→CMap CID+1931, cmap CID+20210

\test{\CID{1931}} \test{\CID{20210}} 

U+2FCA→CMap CID+2055, cmap CID+8717

\test{\CID{2055}} \test{\CID{8717}} 

U+884B→CMap CID+19777, cmap CID+22467

\test{\CID{19777}} \test{\CID{22467}} 

U+F95F→CMap CID+13971, cmap CID+3297

\test{\CID{13971}} \test{\CID{3297}} 

\gtfamily
{\Large ゴシック}

U+00AD→CMap CID+151, cmap CID+14

\test{\CID{151}} \test{\CID{14}} 

U+2F2C→CMap CID+4658, cmap CID+16837

\test{\CID{4658}} \test{\CID{16837}} 

U+2FCA→CMap CID+2055, cmap CID+8717

\test{\CID{2055}} \test{\CID{8717}} 

U+884B→CMap CID+19777, cmap CID+22467

\test{\CID{19777}} \test{\CID{22467}} 

U+F95F→CMap CID+13971, cmap CID+3297

\test{\CID{13971}} \test{\CID{3297}} 

\end{document}
