% -*- coding: utf-8 mode:tex -*-

% 原ノ味フォント 20210130 かな縦組み vpal テスト

\font\f="[HaranoAjiMincho-Medium.otf]"
\f

原ノ味フォント 20210130 かな縦組み vpal テスト

原ノ味の +vert は、小書き文字「ょ」など縦横で異なる必要があるグリフは

源ノの +vert と同じグリフだが、それ以外の縦横で同じでもよい「ち」などは、

源ノの -vert と同じになっている。(形はほぼ同じだが大きさが微妙に異なる模様。)

原ノ味の +vert,+vkna は、源ノの +vert と同じになっている。

原ノ味の +vert,+vkna,+vpal はプロポーショナルになり、原ノ味の
+pwid,+vert とほぼ同じ、

源ノの +vert,+vpal ともほぼ同じに見える。

原ノ味の +vert,+vpal はプロポーショナルにならず、原ノ味
+vert とほぼ同じに見える。

(原ノ味で +vpal のプロポーショナル縦組みするには
+vkna を併用する必要がある。)

\font\f="[HaranoAjiGothic-Regular.otf]:vertical,+vert"
\f
原ノ味角ゴシックReguar

ちちちちちょょょょょ+vert

\font\f="[HaranoAjiGothic-Regular.otf]:vertical,+vert,+vkna"
\f

ちちちちちょょょょょ+vert,+vkna

\font\f="[HaranoAjiGothic-Regular.otf]:vertical,+vert,+vkna,+vpal"
\f
ちちちちちょょょょょ+vert,+vkna,+vpal

\font\f="[HaranoAjiGothic-Regular.otf]:vertical,+vert,+vpal"
\f
ちちちちちょょょょょ+vert,+vpal

\font\f="[HaranoAjiGothic-Regular.otf]:vertical,+pwid,+vert"
\f
ちちちちちょょょょょ+pwid,+vert

\font\f="[HaranoAjiMincho-Regular.otf]:vertical,+vert"
\f
原ノ味明朝Reguar

ちちちちちょょょょょ+vert

\font\f="[HaranoAjiMincho-Regular.otf]:vertical,+vert,+vkna"
\f

ちちちちちょょょょょ+vert,+vkna

\font\f="[HaranoAjiMincho-Regular.otf]:vertical,+vert,+vkna,+vpal"
\f
ちちちちちょょょょょ+vert,+vkna,+vpal

\font\f="[HaranoAjiMincho-Regular.otf]:vertical,+vert,+vpal"
\f
ちちちちちょょょょょ+vert,+vpal

\font\f="[HaranoAjiMincho-Regular.otf]:vertical,+pwid,+vert"
\f
ちちちちちょょょょょ+pwid,+vert

\font\f="[SourceHanSans-Regular.otf]:vertical,+vert"
\f
源ノ角ゴシックReguar

ちちちちちょょょょょ+vert

\font\f="[SourceHanSans-Regular.otf]:vertical,+vert,+vpal"
\f

ちちちちちょょょょょ+vert,+vpal

\font\f="[SourceHanSerif-Regular.otf]:vertical,+vert"
\f
源ノ明朝Reguar

ちちちちちょょょょょ+vert

\font\f="[SourceHanSerif-Regular.otf]:vertical,+vert,+vpal"
\f

ちちちちちょょょょょ+vert,+vpal

\bye
