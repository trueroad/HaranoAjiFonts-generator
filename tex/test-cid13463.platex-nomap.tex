% -*- coding: utf-8 mode:latex -*-

% 原ノ味フォント 20210410 CID+13463 グリフ
% CTAN haranoaji パッケージ収録分

\documentclass[dvipdfmx]{jsarticle}
\usepackage[deluxe]{otf}
\usepackage{lmodern}
\usepackage[noalphabet,otf]{pxchfon}
\usepackage{xcolor}

% haranoaji パッケージ収録の 7 フォントを割り当てる
% (pxchfon のプリセット haranoaji と同じ)
\setminchofont{HaranoAjiMincho-Regular.otf}
\setlightminchofont{HaranoAjiMincho-Light.otf}
\setboldminchofont{HaranoAjiMincho-Bold.otf}
\setgothicfont{HaranoAjiGothic-Medium.otf}
\setmediumgothicfont{HaranoAjiGothic-Regular.otf}
\setboldgothicfont{HaranoAjiGothic-Bold.otf}
\setxboldgothicfont{HaranoAjiGothic-Heavy.otf}
\setmarugothicfont{HaranoAjiGothic-Medium.otf}

\usepackage{test-cid13463}

\title{原ノ味フォント 20210410 CID+13463 グリフ \\
--- CTAN haranoaji パッケージ収録分 ---}
\author{細田 真道}

\begin{document}

\maketitle

\abstract{%
  源ノ角ゴシック2.003版(2021年4月8日リリース)は、
  U+4E08 U+E0101(\CID{13463}、AJ1 CID+13463相当)の
  グリフ(グリフ名uni4E08uE0101-JP)がなくなった。
  これはU+4E08(\CID{2510}、AJ1 CID+2510相当)の
  グリフ(グリフ名uni4E08-CN)と統合されたものと思われる。
  これらは源ノ明朝1.001版(2017年5月8日リリース、
  2021年4月10日現在の最新版)では
  前者がヒゲ付き「\CID{13463}」後者がヒゲ無し「\CID{2510}」となっていて
  グリフが異なる。
  しかし、源ノ角ゴシックの一つ前の版である
  2.002版(2020年11月3日リリース)では、
  前者「{\gtfamily{\CID{13463}}}」後者「{\gtfamily{\CID{2510}}}」ともに
  ヒゲ無しのよく似たグリフになっていた。
  一方、小塚フォントでは、小塚明朝ではヒゲ有無の区別があるが、
  小塚ゴシックでは双方ともヒゲ無しになっているようである。

  源ノ角ゴシック2.002版をベースとした原ノ味角ゴシック20210130版では、
  前者のグリフをCID+13463に、後者のグリフをCID+2510にマッピングしていた。
  しかし、源ノ角ゴシック2.003版では前者のグリフが無くなってしまったので、
  原ノ味角ゴシック20210410版では
  後者のグリフをCID+13463にコピーすることにする。

  本ファイルはCID+13463とCID+2510のグリフを比較するためのものである。
}

\clearpage
\parindent=0pt
\fboxsep=0pt

\mcfamily\ltseries
{\Large 原ノ味明朝\textmd{Light}}

\testAll

\vspace{\baselineskip}

\mcfamily\mdseries
{\Large 原ノ味明朝\textmd{Regular}}

\testAll

\vspace{\baselineskip}

\mcfamily\bfseries
{\Large 原ノ味明朝\textmd{Bold}}

\testAll

\clearpage

\gtfamily\mdseries
{\Large 原ノ味角ゴシック\textmd{Regular}}

\testAll

\vspace{\baselineskip}

\mgfamily % 丸ゴシックの代替が角ゴシックMediumなのでそれを利用
{\Large 原ノ味角ゴシック\textmd{Medium}}

\testAll

\vspace{\baselineskip}

\gtfamily\bfseries
{\Large 原ノ味角ゴシック\textmd{Bold}}

\testAll

\vspace{\baselineskip}

\gtfamily\ebseries
{\Large 原ノ味角ゴシック\textmd{Heavy}}

\testAll

\end{document}
