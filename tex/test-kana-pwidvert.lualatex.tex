% -*- coding: utf-8 mode:latex -*-

% 原ノ味フォント 202012xx プロポーショナルかな 縦組みテスト

\documentclass[tate]{jlreq}

\title{原ノ味フォント202012xxプロポーショナルかな縦組みテスト(失敗)}

\author{細田 真道}

\tfont\hmrnone=HaranoAjiMincho-Regular:jfm=propv;+vert
\tfont\hmrvpal=HaranoAjiMincho-Regular:jfm=propv;+vert;+vpal
\tfont\hmrpwidvert=HaranoAjiMincho-Regular:jfm=propv;+pwid;+vert

\tfont\hgrnone=HaranoAjiGothic-Regular:jfm=propv;+vert
\tfont\hgrvpal=HaranoAjiGothic-Regular:jfm=propv;+vert;+vpal
\tfont\hgrpwidvert=HaranoAjiGothic-Regular:jfm=propv;+pwid;+vert

\tfont\sernone=SourceHanSerif-Regular:jfm=propv;+vert
\tfont\servpal=SourceHanSerif-Regular:jfm=propv;+vert;+vpal
\tfont\serpwidvert=SourceHanSerif-Regular:jfm=propv;+pwid;+vert

\tfont\sarnone=SourceHanSans-Regular:jfm=propv;+vert
\tfont\sarvpal=SourceHanSans-Regular:jfm=propv;+vert;+vpal
\tfont\sarpwidvert=SourceHanSans-Regular:jfm=propv;+pwid;+vert

\begin{document}

\maketitle

\section*{概要}

残念ながら2020年12月xx日現在、本テストはうまく動いてくれない。

原ノ味と源ノで共通のはずの+vpalの動作が共通しておかしいように見える。
どちらもグリフの位置は変更されているが、高さが低くなっていない。

また、原ノ味202012xxで搭載したプロポーショナルかな縦組みは、
+pwid適用後に+vertを適用することでグリフが得られる
(Adobe-Japan1のサイトで配布されているGSUBと同じ)ようになっているが、
選択されたグリフはプロポーショナルかな横組み用になっていた。
つまり、本テストで必要となる
プロポーショナルかな縦組み用を選択できていない
(選択されたプロポーショナルかな横組み用の高さは
  全角と同じなので変わらない)。

源ノは+pwidが無く、+vertのみが適用されるため、問題ないように見える
(もちろんプロポーショナルにはなっていない)。

\clearpage

\section*{原ノ味明朝Regular}

\hmrnone\hbox{\tate\vrule ちょっと\vrule +vert}

\hmrvpal\hbox{\tate\vrule ちょっと\vrule +vert;+vpal}

\hmrpwidvert\hbox{\tate\vrule ちょっと\vrule +pwid;+vert}

\section*{原ノ味角ゴシックRegular}

\hgrnone\hbox{\tate\vrule ちょっと\vrule +vert}

\hgrvpal\hbox{\tate\vrule ちょっと\vrule +vert;+vpal}

\hgrpwidvert\hbox{\tate\vrule ちょっと\vrule +pwid;+vert}

\section*{源ノ明朝Regular}

\sernone\hbox{\tate\vrule ちょっと\vrule +vert}

\servpal\hbox{\tate\vrule ちょっと\vrule +vert;+vpal}

\serpwidvert\hbox{\tate\vrule ちょっと\vrule +pwid;+vert}

\section*{源ノ角ゴシックRegular}

\sarnone\hbox{\tate\vrule ちょっと\vrule +vert}

\sarvpal\hbox{\tate\vrule ちょっと\vrule +vert;+vpal}

\sarpwidvert\hbox{\tate\vrule ちょっと\vrule +pwid;+vert}

\end{document}
