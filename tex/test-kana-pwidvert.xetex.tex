% -*- coding: utf-8 mode:tex -*-

% 原ノ味フォント 20210101 プロポーショナルかな縦組みテスト

\font\f="[HaranoAjiMincho-Medium.otf]"
\f

原ノ味フォント20210101プロポーショナルかな縦組みテスト(失敗)

LuaLaTeX ではプロポーショナルかな縦組みのテストがうまくいかなかったので、

XeTeX でテストしてみる。

XeTeX でも +pwid, +vert によるプロポーショナルかな縦組みグリフ選択は

うまくいかないようなので、 CID 直接指定も用いている。

しかし、CID 直接指定ではプロポーショナルかな縦組みグリフが選択されるが、

余計な文字間スペースが挿入されて高さが全角と変わらないように見える。

(+vpal の場合は通常の全角グリフが選択されるが、
負の文字間スペースが挿入されて、

プロポーショナルに見える。)

\font\hgr="[HaranoAjiGothic-Regular.otf]:vertical,+vert"
\hgr
ちょっと+vert

\font\hgrvpal="[HaranoAjiGothic-Regular.otf]:vertical,+vert,+vpal"
\hgrvpal
ちょっと+vert,+vpal

\font\hgrpwidvert="[HaranoAjiGothic-Regular.otf]:vertical,+pwid,+vert"
\hgrpwidvert
ちょっと+pwid,+vert

\font\hgrdirect="[HaranoAjiGothic-Regular.otf]:vertical"
\hgrdirect
\XeTeXglyph16017%ち
\XeTeXglyph16055%ょ
\XeTeXglyph16019%っ
\XeTeXglyph16024%と
direct

\font\hmr="[HaranoAjiMincho-Regular.otf]:vertical,+vert"
\hmr
ちょっと+vert

\font\hmrvpal="[HaranoAjiMincho-Regular.otf]:vertical,+vert,+vpal"
\hmrvpal
ちょっと+vert,+vpal

\font\hmrpwidvert="[HaranoAjiMincho-Regular.otf]:vertical,+pwid,+vert"
\hmrpwidvert
ちょっと+pwid,+vert

\font\hmrdirect="[HaranoAjiMincho-Regular.otf]:vertical"
\hmrdirect
\XeTeXglyph16017%ち
\XeTeXglyph16055%ょ
\XeTeXglyph16019%っ
\XeTeXglyph16024%と
direct

\bye
