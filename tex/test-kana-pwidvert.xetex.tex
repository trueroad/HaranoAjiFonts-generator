% -*- coding: utf-8 mode:tex -*-

% 原ノ味フォント 20210102 プロポーショナルかな縦組みテスト

\font\f="[HaranoAjiMincho-Medium.otf]"
\f

原ノ味フォント20210102プロポーショナルかな縦組みテスト

LuaLaTeX でのプロポーショナルかな縦組みのテストは +vpal がおかしかったので、

XeTeX でテストしてみる。

2021 年 1 月 2 日時点の XeTeX では +pwid, +vert や direct がおかしいが、

2020 年春頃の XeTeX ならば問題ないように見えるので、その結果を述べる。

+vpal は XeTeX ではうまくいっているように見える。

+pwid, +vert も XeTeX でもうまくいっているように見えるが、

なぜか原ノ味の +vpal とは微妙にズレていて、源ノの +vpal と同じように見える。

(原ノ味の +vpal は小書き以外が横書き用になり +vpal が無いのかもしれない。)

CID 直接指定ではプロポーショナルかな縦組みグリフが選択されているが、

余計な文字間スペースが挿入されて高さが全角と変わらないように見える。

(+vpal の場合は通常の全角グリフが選択されるが、
負の文字間スペースが挿入されて、

プロポーショナルに見える。)

\font\hgr="[HaranoAjiGothic-Regular.otf]:vertical,+vert"
\hgr
原ノ味角ゴシックReguar

ちょっと+vert

\font\hgrvpal="[HaranoAjiGothic-Regular.otf]:vertical,+vert,+vpal"
\hgrvpal
ちょっと+vert,+vpal

\font\hgrpwidvert="[HaranoAjiGothic-Regular.otf]:vertical,+pwid,+vert"
\hgrpwidvert
ちょっと+pwid,+vert

\font\hgrdirect="[HaranoAjiGothic-Regular.otf]:vertical"
\hgrdirect
\XeTeXglyph16017%ち
\XeTeXglyph16055%ょ
\XeTeXglyph16019%っ
\XeTeXglyph16024%と
direct

\font\hmr="[HaranoAjiMincho-Regular.otf]:vertical,+vert"
\hmr
原ノ味明朝Regular

ちょっと+vert

\font\hmrvpal="[HaranoAjiMincho-Regular.otf]:vertical,+vert,+vpal"
\hmrvpal
ちょっと+vert,+vpal

\font\hmrpwidvert="[HaranoAjiMincho-Regular.otf]:vertical,+pwid,+vert"
\hmrpwidvert
ちょっと+pwid,+vert

\font\hmrdirect="[HaranoAjiMincho-Regular.otf]:vertical"
\hmrdirect
\XeTeXglyph16017%ち
\XeTeXglyph16055%ょ
\XeTeXglyph16019%っ
\XeTeXglyph16024%と
direct

\font\sar="[SourceHanSans-Regular.otf]:vertical,+vert"
\sar
源ノ角ゴシックRegular

ちょっと+vert

\font\sarvpal="[SourceHanSans-Regular.otf]:vertical,+vert,+vpal"
\sarvpal
ちょっと+vert,+vpal

\font\sarpwidvert="[SourceHanSans-Regular.otf]:vertical,+pwid,+vert"
\sarpwidvert
ちょっと+pwid,+vert

\font\ser="[SourceHanSerif-Regular.otf]:vertical,+vert"
\ser
源ノ明朝Regular

ちょっと+vert

\font\servpal="[SourceHanSerif-Regular.otf]:vertical,+vert,+vpal"
\servpal
ちょっと+vert,+vpal

\font\serpwidvert="[SourceHanSerif-Regular.otf]:vertical,+pwid,+vert"
\serpwidvert
ちょっと+pwid,+vert

\bye
