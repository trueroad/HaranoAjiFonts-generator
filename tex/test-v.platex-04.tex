% -*- coding: utf-8 mode:latex -*-

%\documentclass{tarticle}
\documentclass[tate]{jlreq}
\usepackage{plext}

\title{縦書きテスト\rensuji{04}字形}
\author{夏目漱石}

\begin{document}

\maketitle

\section*{一}

吾輩は猫である。
名前はまだ無い。

どこで生れたかとんと見当がつかぬ。
何でも薄暗いじめじめした所でニャーニャー泣いていた事だけは記憶している。
吾輩はここで始めて人間というものを見た。

\section*{縦書きグリフ}

\begin{tabular}<t>{|l|l|}
  \hline

  \mcfamily
  明朝
  &
  \gtfamily\sffamily
  ゴシック
  \\

  \hline

  \mcfamily
  読点、句点。
  三点…二点‥
  「カギかっこ」
  (丸カッコ)
  &
  \gtfamily\sffamily
  読点、句点。
  三点…二点‥
  「カギかっこ」
  (丸カッコ)
  \\

  \mcfamily
  【‖‖‖‖‖】一区三十四点U+2016『双柱』
  &
  \gtfamily\sffamily
  【‖‖‖‖‖】一区三十四点U+2016『双柱』
  \\

  \mcfamily
  【°°°°°】一区七十五点U+00B0『度』
  &
  \gtfamily\sffamily
  【°°°°°】一区七十五点U+00B0『度』
  \\

  \mcfamily
  【′′′′′】一区七十六点U+2032『分』
  &
  \gtfamily\sffamily
  【′′′′′】一区七十六点U+2032『分』
  \\

  \mcfamily
  【″″″″″】一区七十七点U+2033『秒』
  &
  \gtfamily\sffamily
  【″″″″″】一区七十七点U+2033『秒』
  \\

  \hline
\end{tabular}

\end{document}
