% -*- coding: utf-8 mode:latex -*-

% 原ノ味フォント全角かなグリフ一覧
% CTAN haranoaji パッケージ収録分

\documentclass[dvipdfmx]{jsarticle}
\usepackage[deluxe]{otf}
\usepackage{lmodern}
\usepackage[noalphabet,otf]{pxchfon}
\usepackage{xcolor}

% haranoaji パッケージ収録の 7 フォントを割り当てる
% (pxchfon のプリセット haranoaji と同じ)
\setminchofont{HaranoAjiMincho-Regular.otf}
\setlightminchofont{HaranoAjiMincho-Light.otf}
\setboldminchofont{HaranoAjiMincho-Bold.otf}
\setgothicfont{HaranoAjiGothic-Medium.otf}
\setmediumgothicfont{HaranoAjiGothic-Regular.otf}
\setboldgothicfont{HaranoAjiGothic-Bold.otf}
\setxboldgothicfont{HaranoAjiGothic-Heavy.otf}
\setmarugothicfont{HaranoAjiGothic-Medium.otf}

\usepackage{test-kana-fwid}

\title{原ノ味フォント全角かなグリフ一覧\\
--- CTAN haranoaji パッケージ収録分 ---}
\author{細田 真道}

\begin{document}

\maketitle

\abstract{%
  原ノ味フォント20210101版は、
  横組み用のCID+12269小書き「こ」およびCID+12271小書き「コ」、
  縦組み用のCID+12270小書き「こ」およびCID+12272小書き「コ」
  を搭載していない。
  これは、源ノフォントがこれらのグリフを搭載していないためである。
}

\clearpage
\parindent=0pt
\fboxsep=0pt

\mcfamily\ltseries
{\huge 原ノ味明朝\textmd{Light}}

\testAll

\clearpage

\mcfamily\mdseries
{\huge 原ノ味明朝\textmd{Regular}}

\testAll

\clearpage

\mcfamily\bfseries
{\huge 原ノ味明朝\textmd{Bold}}

\testAll

\clearpage

\gtfamily\mdseries
{\huge 原ノ味角ゴシック\textmd{Regular}}

\testAll

\clearpage

\mgfamily % 丸ゴシックの代替が角ゴシックMediumなのでそれを利用
{\huge 原ノ味角ゴシック\textmd{Medium}}

\testAll

\clearpage

\gtfamily\bfseries
{\huge 原ノ味角ゴシック\textmd{Bold}}

\testAll

\clearpage

\gtfamily\ebseries
{\huge 原ノ味角ゴシック\textmd{Heavy}}

\testAll

\end{document}
