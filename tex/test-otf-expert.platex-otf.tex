\documentclass{jsarticle}
\usepackage[expert]{otf}
\usepackage{okumacro}

\rubyfamily

\begin{document}

\section{OTFパッケージexpertオプション}

このファイルは\ruby{原ノ味}{はらのあじ}フォントのダミーグリフが
\ruby{正常}{せいじょう}に\ruby{生成}{せいせい}されていることを
\ruby{確認}{かくにん}するためのものである。
そのためにOTFパッケージを
expertオプション\ruby{付}{つ}きにした上で
\textbackslash rubyfamilyを\ruby{使用}{しよう}し、
\ruby{同}{おな}じDVIから、\ruby{原ノ味}{はらのあじ}フォント
\ruby{埋}{う}め\ruby{込}{こ}みPDFと
フォント\ruby{非}{ひ}\ruby{埋}{う}め\ruby{込}{こ}みPDFの
\ruby{双方}{そうほう}を\ruby{生成}{せいせい}している。
\ruby{原ノ味}{はらのあじ}フォント202012xx版は、
expertオプションで\ruby{使}{つか}われる、
\ruby{組方向}{くみほうこう}に\ruby{応}{おう}じた\ruby{専用}{せんよう}かなを
\ruby{持}{も}っているが、
ルビ\ruby{用}{よう}かなは\ruby{持}{も}っていないため、
\ruby{原ノ味}{はらのあじ}フォント\ruby{埋}{う}め\ruby{込}{こ}みPDFでは
ダミーグリフ(「\CID{0}」の形)で\ruby{表示}{ひょうじ}される。
\ruby{小塚}{こづか}フォントを\ruby{持}{も}つAcrobat Readerで
フォント\ruby{非}{ひ}\ruby{埋}{う}め\ruby{込}{こ}みPDFを
\ruby{開}{ひら}くと、ダミーグリフではない\ruby{本来}{ほんらい}の
\ruby{表示}{ひょうじ}が\ruby{確認}{かくにん}できる。

\end{document}
