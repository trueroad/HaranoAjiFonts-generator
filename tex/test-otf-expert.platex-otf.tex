\documentclass{jsarticle}
\usepackage[expert]{otf}
\usepackage{okumacro}

\rubyfamily

\begin{document}

\section{OTFパッケージexpertオプション}

このファイルは\ruby{原ノ味}{はらのあじ}フォントのダミーグリフが
\ruby{正常}{せいじょう}に\ruby{生成}{せいせい}されていることを
\ruby{確認}{かくにん}するためのものである。
そのためにOTFパッケージを
expertオプション\ruby{付}{つ}きで\ruby{使用}{しよう}し、
\ruby{同}{おな}じDVIから、\ruby{原ノ味}{はらのあじ}フォント
\ruby{埋}{う}め\ruby{込}{こ}みPDFと
フォント\ruby{非}{ひ}\ruby{埋}{う}め\ruby{込}{こ}みPDFの
\ruby{双方}{そうほう}を\ruby{生成}{せいせい}している。
\ruby{原ノ味}{はらのあじ}フォントは、
expertオプションで\ruby{使}{つか}われる、
\ruby{組方向}{くみほうこう}に\ruby{応}{おう}じた\ruby{専用}{せんよう}かなや、
ルビ\ruby{用}{よう}かなを\ruby{持}{も}っていないため、
\ruby{原ノ味}{はらのあじ}フォント\ruby{埋}{う}め\ruby{込}{こ}みPDFでは
ダミーグリフで\ruby{表示}{ひょうじ}される。
\ruby{小塚}{こづか}フォントを\ruby{持}{も}つAcrobat Readerで
フォント\ruby{非}{ひ}\ruby{埋}{う}め\ruby{込}{こ}みPDFを
\ruby{開}{ひら}くと、ダミーグリフではない\ruby{本来}{ほんらい}の
\ruby{表示}{ひょうじ}が\ruby{確認}{かくにん}できる。

Note: If you use HaranoAji Fonts,
all HIRAGANA and KATAKANA glyphs are displayed as DUMMY glyphs (i.e. \CID{0})
in this document.

\end{document}
